\documentclass[11pt,a4paper]{article}

\usepackage[utf8]{inputenc}
\usepackage[francais]{babel}
\usepackage[T1]{fontenc}

%default values :
%\textwidth = 390pt
%\textheight = 592pt
\usepackage[textwidth=460pt, textheight=670pt]{geometry}
\pagenumbering{gobble}

%blinded footnote command
\newcommand\blfootnote[1]{%
  \begingroup
  \renewcommand\thefootnote{}\footnote{#1}%
  \addtocounter{footnote}{-1}%
  \endgroup
}

\begin{document}

\section*{Manifeste pour le développement Agile de logiciels}

Nous découvrons comment mieux développer des logiciels
par la pratique et en aidant les autres à le faire.
\\Ces expériences nous ont amenés à valoriser :
\vspace{5mm}

{\LARGE Les individus et leurs interactions} {\large plus que les processus et les outils}

{\LARGE Des logiciels opérationnels} {\large plus qu’une documentation exhaustive}

{\LARGE La collaboration avec les clients} {\large plus que la négociation contractuelle}

{\LARGE L’adaptation au changement} {\large plus que le suivi d’un plan}
\vspace{5mm}
~\\Nous reconnaissons la valeur des seconds éléments,
mais privilégions les premiers.


\subsection*{Principes sous-jacents au manifeste}

Nous suivons ces principes:
\begin{itemize}
\item[$\circ$] Notre plus haute priorité est de satisfaire le client
en livrant rapidement et régulièrement des fonctionnalités
à grande valeur ajoutée.
\item[$\circ$] Accueillez positivement les changements de besoins,
même tard dans le projet. Les processus Agiles
exploitent le changement pour donner un avantage
compétitif au client.
\item[$\circ$] Livrez fréquemment un logiciel opérationnel avec des
cycles de quelques semaines à quelques mois et une
préférence pour les plus courts.
\item[$\circ$] Les utilisateurs ou leurs représentants et les
développeurs doivent travailler ensemble quotidiennement
tout au long du projet.
\item[$\circ$] Réalisez les projets avec des personnes motivées.
Fournissez-leur l’environnement et le soutien dont ils
ont besoin et faites-leur confiance pour atteindre les
objectifs fixés.
\item[$\circ$] La méthode la plus simple et la plus efficace pour
transmettre de l’information à l'équipe de développement
et à l’intérieur de celle-ci est le dialogue en face à face.
\item[$\circ$] Un logiciel opérationnel est la principale mesure d’avancement.
\item[$\circ$] Les processus Agiles encouragent un rythme de développement
soutenable. Ensemble, les commanditaires, les développeurs
et les utilisateurs devraient être capables de maintenir
indéfiniment un rythme constant.
\item[$\circ$] Une attention continue à l'excellence technique et
à une bonne conception renforce l’Agilité.
\item[$\circ$] La simplicité -- c’est-à-dire l’art de minimiser la
quantité de travail inutile -- est essentielle.
\item[$\circ$] Les meilleures architectures, spécifications et
conceptions émergent d'équipes auto-organisées.
\item[$\circ$] À intervalles réguliers, l'équipe réfléchit aux moyens
de devenir plus efficace, puis règle et modifie son
comportement en conséquence.
\end{itemize}


\vspace{5mm}


\begin{center}
\begin{tabular}{ccc}
Kent Beck & James Grenning & Robert C. Martin \\ 
Mike Beedle & Jim Highsmith & Steve Mellor \\ 
Arie van Bennekum & Andrew Hunt & Ken Schwaber \\ 
Alistair Cockburn & Ron Jeffries & Jeff Sutherland \\ 
Ward Cunningham & Jon Kern & Dave Thomas \\ 
Martin Fowler & Brian Marick & ~ \\ 
\end{tabular} 
\end{center}

\blfootnote{\copyright~2001, the above authors.\\This declaration may be freely copied in any form, but only in its entirety through this notice.}

\end{document}
